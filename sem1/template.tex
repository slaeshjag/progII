%% En enkel mall för att skapa en labb-raport.
\documentclass[a4paper, 11pt]{article}
\usepackage[utf8]{inputenc} 
\usepackage[swedish]{babel}
\usepackage{textcomp}

\title{Seminarie 1}
\author{Steven Arnow}
\date{\today} 

\begin{document}

\maketitle 

\section{Uppgiften}

\subsection{Reverse}

Uppgiften går ut på att jämföra prestandan hos två implementationer av en funktion som vänder på en lista, så att sista elementet hamnar först etc..

\subsection{Fibonacci}
Uppgiften går ut på att konstruera ett funktionsanrop som beräknar n:te fibonaccitalet. Eftersom en funktion som strikt var rekursiv till sig själv med formatet \texttt{fib(n)/1} skulle ta hundra år på sig att beräkna ett tvåsiffrigt fibonaccital, valde jag att enbart använda den funktionen för att bootstrappa en annan rekursiv funktion.

Vad går uppgiften ut på, beskriv kortfattat problemet och hur det skulle lösas.






\section{Ansatts}

\subsection{Fibonacci}

För små tal är tiden det tar ganska linjär och förutsägbar. Som nedan kan ses i koden nedan så har jag inget tålamod för rakt av efterblivna implementeringar som tar år och dagar på att göra samma sak hela tiden.

\begin{verbatim}
fib(M, N, C, T) -> case C < T of
        true -> fib(N, M + N, C + 1, T);
        false -> N
end.

fib(N) -> fib(0, 1, 0, N - 1).
\end{verbatim}

Baserat på koden och testkörningar med små fibonaccital förväntar jag mig att beräkningstiden bör öka någorlunda linjärt. Hur det blir med overhead när resultatvärdet växer kan jag dock inte sia om.






\section{Utvärdering}

\subsection{reverse}

Uppmätt körtid i tabell \ref{tab:reverse} visar att, som hintat i uppgiften, att \emph{reverse} är rejalt mycket snabbare än \emph{nreverse}. Komplexiteten hos \emph{nreverse} är nästan kvadratisk ($O(n^2)$,) jämfört med \emph{reverse} som närmast ser linjär ut ($O(n)$), där \emph{n} är längden på listan som ska vändas. Detta beror främst på att eftersom \emph{reverse} reäknar ut allt innan nästa rekursion, vilket betyder att den kan optimeras med svansrekursion. Den bygger dessutom bara på en lista, i stället för att skapa en ny varje rekursion. Att skapa en ny lista är långsamt, speciellt om den skall appendas varje rekursion.

\begin{table}
\centering
\begin{tabular}{|l|r|r|}
\hline
Antal element & Tid för reverse (µs) & Tid för nreverse (µS)\\
\hline
16 & 178 & 26 \\
\hline
32 & 729 & 48 \\
\hline
64 & 1850 & 72 \\
\hline
128 & 7962 & 132 \\
\hline
256 & 26360 & 256 \\
\hline
512 & 105701 & 417 \\
\hline
\end{tabular}
\caption{Beräkningstid vs. implementation vs. listlängd}
\label{tab:reverse}
\end{table}




\subsection{Fibonacci}

Uppmätt beräkningstid i tabell \ref{tab:fibobench} visar att upp till någonstans mellan fibonaccital 10000 och 100000 börjar overhead öka drastiskt.

\begin{table}
\centering
\begin{tabular}{|l|r|r|}
\hline
Fibonacci \textnumero & Beräkningstid (µS) & µS/\textnumero\\
\hline
10 & 13 & 1.3\\
\hline
100 & 137 & 1.37\\
\hline
1000 & 1668 & 1.67\\
\hline
10000 & 24372 & 2.44\\
\hline
100000 & 2280810 & 22.8\\
\hline
500000 & 49261793 & 98.5\\
\hline
\end{tabular}
\caption{Beräkningstid vs. storlek på fibonaccitalet}
\label{tab:fibobench}
\end{table}

Jag kom fram till tidsapproximationen $t = n \times (1.3 + \frac{n}{5050})$ för min lösning på min laptop. När jag skriver den här delen av rapporten är det en vecka kvar till seminariet, vilket enligt approximationen ger att jag borde kunna räkna ut ett fibonaccital någonstans i krokarna runt 54000000.

Uppgifterna i denna kurs kanske inte kräver en utvärdering men man kan
här visa resultat från testkörningar mm. Om man vill lägga in resultat
från testkörningar kan man sammanfatta dessa i en tabell. Ett exempel
kan ses i tabell \ref{tab:results}. 



\section{Sammanfattning}


Vad gick bra och vad gick mindre bra? Vad var de största problemen och
hur löstes de? Skriv en kort sammanfattning.

\end{document}
