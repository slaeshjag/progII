%% En enkel mall för att skapa en labb-raport.
\documentclass[a4paper, 11pt]{article}
\usepackage[utf8]{inputenc} 
\usepackage[swedish]{babel}
\usepackage{textcomp}

\title{Seminarie 1}
\author{Steven Arnow}
\date{\today} 

\begin{document}

\maketitle 

\section{Uppgiften}

\subsection{Reverse}

Uppgiften går ut på att jämföra prestandan hos två implementationer av en funktion som vänder på en lista, så att sista elementet hamnar först etc..


\subsection{Binärkodning}

Uppgiften går ut på att konstruera två olika implementationer av en funktion som tar ett heltal som indata, och returnerar en bitrepresentation av talet i form av en lista, med en etta eller nolla per element.


\subsection{Fibonacci}
Uppgiften går ut på att konstruera ett funktionsanrop som beräknar n:te fibonaccitalet. Eftersom en funktion som strikt var rekursiv till sig själv med formatet \texttt{fib(n)/1} skulle ta hundra år på sig att beräkna ett tvåsiffrigt fibonaccital, valde jag att enbart använda den funktionen för att bootstrappa en annan rekursiv funktion.






\section{Ansatts}

\subsection{Binärkodning}

Ena implementationen jag gjorde var att rekursivt iterera över startnumret, där en division görs med talet motsvarande bitpositionen som skall plockas ut, och resten jämförs med positionen under biten som testades. Är resten större eller lika med värdet för positionen under så är det en etta på en positionen, annars en nolla. Sedan delas positionsvärde med två och rekurserar vidare tills testade positionsvärdet är ett. Kod nedan.

\begin{verbatim}
get_bit(Num, Pos) -> case Pos =< 1 of
        true -> [];
        false -> (case (Num rem Pos) >= (Pos div 2) of
                true -> [1|get_bit(Num, Pos div 2)];
                false -> [0|get_bit(Num, Pos div 2)]
        end)
end.

find_pos(Num, Pos -> case Pos =< Num of
        true -> find_pos(Num, Pos * 2);
        false -> get_bit(Num, Pos)
end.

get_bit(Num) -> find_pos(Num, 2).
\end{verbatim}

Mycket kod. Mycket mer än min andra implementation, som använder en standardfunktion för att göra om ett heltal till en textsträng, som beskriver talet i angiven bas i ASCII. Därefter applicerade jag en rekursionsloop som subtraherar 48 (ASCII-värdet för '0') så att listan ser ut som det frågades efter.

\begin{verbatim}
list_unchr(Lst) -> case Lst == [] of
        true -> [];
        false -> [hd(Lst) - 48 | list_unchr(tl(Lst))]
end.
get_bit2(Num) -> list_unchr(integer_to_list(Num, 2)).
\end{verbatim}

\subsection{Fibonacci}

För små tal är tiden det tar ganska linjär och förutsägbar. Som nedan kan ses i koden nedan så har jag inget tålamod för rakt av efterblivna implementeringar som tar år och dagar på att göra samma sak hela tiden. Jag valde att lösa det genom att jobba uppåt hela tiden, och holla koll på vilket fibonaccital som räknas ut geom ett räkneargument. När rätt fibonaccital hittats kollapsar rekursionen ihop och det beräknade talet returneras ner. Eftersom inga berökningar görs efter rekursionen, är dessutom funktionen svansrekursiv.

\begin{verbatim}
fib(M, N, C, T) -> case C < T of
        true -> fib(N, M + N, C + 1, T);
        false -> N
end.

fib(N) -> fib(0, 1, 0, N - 1).
\end{verbatim}

Baserat på koden och testkörningar med små fibonaccital förväntar jag mig att beräkningstiden bör öka någorlunda linjärt. Hur det blir med overhead när resultatvärdet växer kan jag dock inte sia om.






\section{Utvärdering}

\subsection{reverse}

Uppmätt körtid i tabell \ref{tab:reverse} visar att, som hintat i uppgiften, att \emph{reverse} är rejalt mycket snabbare än \emph{nreverse}. Komplexiteten hos \emph{nreverse} är nästan kvadratisk ($O(n^2)$,) jämfört med \emph{reverse} som närmast ser linjär ut ($O(n)$), där \emph{n} är längden på listan som ska vändas. Detta beror främst på att eftersom \emph{reverse} reäknar ut allt innan nästa rekursion, vilket betyder att den kan optimeras med svansrekursion. Den bygger dessutom bara på en lista, i stället för att skapa en ny varje rekursion. Att skapa en ny lista är långsamt, speciellt om den skall appendas varje rekursion.

\begin{table}
\centering
\begin{tabular}{|l|r|r|}
\hline
Antal element & Tid för reverse (µs) & Tid för nreverse (µS)\\
\hline
16 & 178 & 26 \\
\hline
32 & 729 & 48 \\
\hline
64 & 1850 & 72 \\
\hline
128 & 7962 & 132 \\
\hline
256 & 26360 & 256 \\
\hline
512 & 105701 & 417 \\
\hline
\end{tabular}
\caption{Beräkningstid vs. implementation vs. listlängd}
\label{tab:reverse}
\end{table}



\subsection{Binärkodning}

Av de två implementationer är den senare av de två den bästa. Ingen av dom är bra, men den senare är bättre än den första... Dels är den senare mindre kod, och dels är den senare antagligen snabbare, eftersom grovjobbet görs av en funktion i erlangs standardbibliotek, som förhoppningsvis är skriven i ett snabbare språk än erlang..



\subsection{Fibonacci}

Uppmätt beräkningstid i tabell \ref{tab:fibobench} visar att upp till någonstans mellan fibonaccital 10000 och 100000 börjar overhead öka drastiskt.

\begin{table}
\centering
\begin{tabular}{|l|r|r|}
\hline
Fibonacci \textnumero & Beräkningstid (µS) & µS/\textnumero\\
\hline
10 & 13 & 1.3\\
\hline
100 & 137 & 1.37\\
\hline
1000 & 1668 & 1.67\\
\hline
10000 & 24372 & 2.44\\
\hline
100000 & 2280810 & 22.8\\
\hline
500000 & 49261793 & 98.5\\
\hline
\end{tabular}
\caption{Beräkningstid vs. storlek på fibonaccitalet}
\label{tab:fibobench}
\end{table}

Jag kom fram till tidsapproximationen $t = n \times (1.3 + \frac{n}{5050})$ för min lösning på min laptop. När jag skriver den här delen av rapporten är det en vecka kvar till seminariet, vilket enligt approximationen ger att jag borde kunna räkna ut ett fibonaccital någonstans i krokarna runt 54000000.



\section{Sammanfattning}


Största problemen jag hade var att brottas med Erlangs syntax och att vada runt i Erlangs biblioteksdokumentation i hopp om att hitta den där magiska funktionen som gör den sökta implementationen snäppet mindre efterbliven. Sett till implementationerna så hade jag hela tiden en klar bild av vad som behövde åstakommas, så det var bara att gräva runt i legolådan tills rimliga bitar hittades.

\end{document}
