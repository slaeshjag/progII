%% En enkel mall för att skapa en labb-raport.
\documentclass[a4paper, 11pt]{article}
\usepackage[utf8]{inputenc} 
\usepackage[swedish]{babel}

\title{Seminarie 2}
\author{Steven Arnow}
\date{\today} 

\begin{document}

\maketitle 

\section{Uppgiften}

Uppgiften går ut på att implementera en huffmankodare/avkodare, samt kod för att generera kodträdet som används av de två. Dessutom ska implementationen benchmarkas. Uppgiften skall lösas genom att göra frekvensanalys av indata, för att konstruera ett träd som kodar förekommande tecken för optimal komprimering. 

\section{Ansatts}

Jag implementerade detta genom att först skapa en sorterad lista av indata, och med hjälp av den skapa en lista med förekomsten av alla möjliga bytes. Med frekvenslistan skapade jag sedan en lista med löv till det kommande trädet. Med lövlistan och frekvenslistan itererar jag sedan över löven, och grupperar upp dom i två löv som sitter fast i en gren. Jag skapar en lista av dessa grenar. När alla löv tagits om hand och sitter fast på en gren sorteras utlistan och matas in som indata på samma sätt som löven gjordes, till samma funktion, som då repeterar processen tills ett enda träd återstår.

Därefter skapas en lista med koder för varje möjlig byte utifrån trädet, detta görs genom att bygga upp en osorterad lista vartefter trädet rekursivt utforskas. När rekursionen faller tillbaka byggs en lista upp av vägarna som togs för att hitta varje löv. I och med hur listan byggs upp hamnar dessutom de vanligast förekommande elementen först.

Med hjälp av listan kan indata lätt kodas efter huffmanträdet, och med huffmanträdet kan utdata snabbt konverteras tillbaka till vad det ursprungligen var som komprimerades.

\section{Utvärdering}


Uppgifterna i denna kurs kanske inte kräver en utvärdering men man kan
här visa resultat från testkörningar mm. Om man vill lägga in resultat
från testkörningar kan man sammanfatta dessa i en tabell. Ett exempel
kan ses i tabell \ref{tab:results}. 


\begin{table}
\centering
\begin{tabular}{|l|r|r|}  
\hline
kärnor & tid & uppsnabbning\\
\hline
1 & 400ms & 1\\
\hline
2 & 240ms & 1.7\\
\hline
4 & 140ms & 2.8\\
\hline
\end{tabular}
\caption{Ha alltid en rad som förklarar vad tabellen handlar om.}
\label{tab:results}
\end{table}

Om ni vill ha med en graf så rekommenderas ni att använda gnuplot. Om
ni lär er använda det för att göra enkla diagram så kommer ni ha
mycket nytta av det i framtiden.


\section{Sammanfattning}

Vad gick bra och vad gick mindre bra? Vad var de största problemen och
hur löstes de? Skriv en kort sammanfattning.

\end{document}
